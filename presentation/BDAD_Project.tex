%%%%%%%%%%%%%%%%%%%%%%%%%%%%%%%%%%%%%%%%%%%%%%%%%%%%%%%%%%%%%%%%%%%%%%%%%%%
%
% BDAD 2019 
% Cody Gilbert, Fang Han, Jeremy Lao
% 
%%%%%%%%%%%%%%%%%%%%%%%%%%%%%%%%%%%%%%%%%%%%%%%%%%%%%%%%%%%%%%%%%%%%%%%%%%%

\documentclass[11pt]{article}

% AMS packages:
\usepackage[fleqn]{amsmath}
\usepackage{amsmath, amsthm, amsfonts}

% Document Formatting Packages:
\usepackage{hyperref}
\usepackage[margin=1in]{geometry}
\usepackage{graphicx}
\usepackage{caption}
\usepackage{subcaption}
\usepackage{float}
\newcommand{\vertSpace}[1]{\vspace{3mm}}


%----------------------------------------------------------------
\title{Using Big Data Systems to Analyze Big (not Jumbo) Mortgage Data}
\author{
        Cody Gilbert \\
        NYU Computer Science \\
            \and
        Fang Han\\
        NYU Computer Science \\
             \and
        Jeremy Lao \\
        NYU Computer Science
}

\begin{document}
{\setlength{\mathindent}{0cm}
\maketitle

\abstract{\textit{abstract here}}

\section{Introduction}


\vertSpace

 \section{Objectives}


\section{Methodology}

\subsection{Data}





\subsection{Apply ML to Sparse Matrices generated from CountVectorizer}



\subsubsection{Training, Testing, and Determining the Efficacy of the Models}



\section{Fruther Work}

\section{Conclusion}
Reducing the sparsity of the matrices generated from CountVectorizer by stacking the documents greatly improved the results.  \vertSpace




\appendix
\section{Appendix}

\subsection{What is HMDA}


% Bibliography
%-----------------------------------------------------------------
\begin{thebibliography}{99}

\bibitem item


\end{thebibliography}
\end{document}
